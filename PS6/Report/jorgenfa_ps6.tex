%%%%%%%%%%%%%%%%%%%%%%%%%%%%%%%%%%%%%%%%%
% Short Sectioned Assignment
% LaTeX Template
% Version 1.0 (5/5/12)
%
% This template has been downloaded from:
% http://www.LaTeXTemplates.com
%
% Original author:
% Frits Wenneker (http://www.howtotex.com)
%
% License:
% CC BY-NC-SA 3.0 (http://creativecommons.org/licenses/by-nc-sa/3.0/)
%
%%%%%%%%%%%%%%%%%%%%%%%%%%%%%%%%%%%%%%%%%

%----------------------------------------------------------------------------------------
%	PACKAGES AND OTHER DOCUMENT CONFIGURATIONS
%----------------------------------------------------------------------------------------

\documentclass[paper=a4, fontsize=11pt]{scrartcl} % A4 paper and 11pt font size

\usepackage[T1]{fontenc} % Use 8-bit encoding that has 256 glyphs
\usepackage{fourier} % Use the Adobe Utopia font for the document - comment this line to return to the LaTeX default
\usepackage[english]{babel} % English language/hyphenation
\usepackage{amsmath,amsfonts,amsthm} % Math packages


\usepackage{sectsty} % Allows customizing section commands
\allsectionsfont{\centering \normalfont\scshape} % Make all sections centered, the default font and small caps

\usepackage{fancyhdr} % Custom headers and footers
\pagestyle{fancyplain} % Makes all pages in the document conform to the custom headers and footers
\fancyhead{} % No page header - if you want one, create it in the same way as the footers below
\fancyfoot[L]{} % Empty left footer
\fancyfoot[C]{} % Empty center footer
\fancyfoot[R]{\thepage} % Page numbering for right footer
\renewcommand{\headrulewidth}{0pt} % Remove header underlines
\renewcommand{\footrulewidth}{0pt} % Remove footer underlines
\setlength{\headheight}{13.6pt} % Customize the height of the header

\renewcommand{\thesection}{\hspace*{-1.0em}} % turn off numbering for section, subsection
\renewcommand{\thesubsection}{\hspace*{-1.0em}}
\numberwithin{equation}{section} % Number equations within sections (i.e. 1.1, 1.2, 2.1, 2.2 instead of 1, 2, 3, 4)
\numberwithin{figure}{section} % Number figures within sections (i.e. 1.1, 1.2, 2.1, 2.2 instead of 1, 2, 3, 4)
\numberwithin{table}{section} % Number tables within sections (i.e. 1.1, 1.2, 2.1, 2.2 instead of 1, 2, 3, 4)

\setlength\parindent{0pt} % Removes all indentation from paragraphs - comment this line for an assignment with lots of text

%----------------------------------------------------------------------------------------
%	TITLE SECTION
%----------------------------------------------------------------------------------------

\newcommand{\horrule}[1]{\rule{\linewidth}{#1}} % Create horizontal rule command with 1 argument of height

\title{	
\normalfont \normalsize 
\textsc{Norwegian University of Science and Technology} \\ [25pt] % Your university, school and/or department name(s)
\horrule{0.5pt} \\[0.4cm] % Thin top horizontal rule
\huge Problem Set 6 \\ % The assignment title
\horrule{2pt} \\[0.5cm] % Thick bottom horizontal rule
}

\author{J{\o}rgen Faret} % Your name

\date{\normalsize\today} % Today's date or a custom date

\begin{document}

\maketitle % Print the title

\section{Problem 1, OpenCL}
\subsection{a)}

\begin{enumerate}
\item CUDA is created by nVidia for nVidia GPUs, while OpenCL is system-independant. OpenCL can be run on multiple platforms including GPUs, CPUs and FPGAs. CUDA can therefore only be used to perform homogeneous computing on GPUs, while OpenCL can be used to perform heterogeneous computing because it can target any device with an input and output. 

\item While both device code and host code is compiled at compile-time in CUDA, device code is compiled at run-time in OpenCL.  

\item CUDA is slightly faster (up to 10\% according to Matt Harvey, developer of Cuda2OpenCL-translator Swan) on nVidia GPUs than OpenCL.
\end{enumerate}

\subsection{b)}

The OpenCL equivalent terms for the CUDA terms thread, block and grid are work item, work-group and grid respectively. 

\subsection{c)}

There is no equivalent term for warp in OpenCL because OpenCL is system-independant, which means the device running the OpenCL program may not even be able to perform parallel instructions. OpenCL is however typically run on devices with SIMD width greater than one. To see what the native hardware execution width is, one can query the kernel object with the function:
\verb!CL_KERNEL_PREFERRED_WORK_GROUP_SIZE_MULTIPLE()! 

\section{2, Graphics}

Computations where the output is an image are very well suited to GPU computations because images are arrays of pixels. Computing the values of these pixels is an easily parallelizable task since these computations are independant, and often not very complex. 

\section{3, Heterogeneous computing}

\subsection{a)}



\end{document}